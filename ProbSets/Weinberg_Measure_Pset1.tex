\documentclass[letterpaper,12pt]{article}
\usepackage{array}
\usepackage{threeparttable}
\usepackage{geometry}
\geometry{letterpaper,tmargin=1in,bmargin=1in,lmargin=1.25in,rmargin=1.25in}
\usepackage{fancyhdr,lastpage}
\pagestyle{fancy}
\lhead{}
\chead{}
\rhead{}
\lfoot{}
\cfoot{}
\rfoot{\footnotesize\textsl{Page \thepage\ of \pageref{LastPage}}}
\renewcommand\headrulewidth{0pt}
\renewcommand\footrulewidth{0pt}
\usepackage[format=hang,font=normalsize,labelfont=bf]{caption}
\usepackage{listings}
\lstset{frame=single,
  showstringspaces=false,
  columns=flexible,
  basicstyle={\small\ttfamily},
  numbers=none,
  breaklines=true,
  breakatwhitespace=true
  tabsize=3
}
\usepackage{amsmath}
\usepackage{amssymb}
\usepackage{amsthm}
\usepackage{harvard}
\usepackage{setspace}
\usepackage{float,color}
\usepackage[pdftex]{graphicx}
\usepackage{hyperref}
\hypersetup{colorlinks,linkcolor=red,urlcolor=blue}
\theoremstyle{definition}
\newtheorem{theorem}{Theorem}
\newtheorem{acknowledgement}[theorem]{Acknowledgement}
\newtheorem{algorithm}[theorem]{Algorithm}
\newtheorem{axiom}[theorem]{Axiom}
\newtheorem{case}[theorem]{Case}
\newtheorem{claim}[theorem]{Claim}
\newtheorem{conclusion}[theorem]{Conclusion}
\newtheorem{condition}[theorem]{Condition}
\newtheorem{conjecture}[theorem]{Conjecture}
\newtheorem{corollary}[theorem]{Corollary}
\newtheorem{criterion}[theorem]{Criterion}
\newtheorem{definition}[theorem]{Definition}
\newtheorem{derivation}{Derivation} % Number derivations on their own
\newtheorem{example}[theorem]{Example}
\newtheorem{exercise}[theorem]{Exercise}
\newtheorem{lemma}[theorem]{Lemma}
\newtheorem{notation}[theorem]{Notation}
\newtheorem{problem}[theorem]{Problem}
\newtheorem{proposition}{Proposition} % Number propositions on their own
\newtheorem{remark}[theorem]{Remark}
\newtheorem{solution}[theorem]{Solution}
\newtheorem{summary}[theorem]{Summary}
%\numberwithin{equation}{section}
\bibliographystyle{aer}
\newcommand\ve{\varepsilon}
\newcommand\boldline{\arrayrulewidth{1pt}\hline}


\begin{document}

\begin{flushleft}
  \textbf{\large{Problem Set \#1}} \\
  Measure Theory, Jan Ertl \\
  Alex Weinberg\footnote{worked with Arpan Chakrabarti and Arushi Saksena}
\end{flushleft}

\vspace{5mm}

\begin{solution}(1.3)
  \begin{enumerate}
  \item
  Suppose $s \in \mathbb{R} $ open. So $s \in \mathbb{G}_1$ by definition. Take $s^c \in \mathbb{R}$ is close by properties of open/closed sets. \\
  $\Rightarrow$ $s^c \not \in \mathbb{G}_1$ by definition of $\mathbb{G}_1$ \\
  $\Rightarrow$ not closed under complements. So G1 is not a sigma-algebra.

  \item \textbf{WTS:} G2 algebra
  \begin{itemize}
  \item $\varnothing \in (a,b] \Rightarrow \varnothing \in \mathbb{G}_2 $
  \item Suppose $A_j = \cup_{i=1}^{N_j} ((a_i,b_i] \cup (-\infty,b] \cup (a,\infty))$ \\
  So then $\cup_j^M A_j = \cup_j^M (\cup_{i=1}^{N_j} ((a_i,b_i] \cup (-\infty,b] \cup (a,\infty))) \in \mathbb{G}_2 \qed$
  \end{itemize}
  So G2 is an algebra.

  \item \textbf{WTS:} G3 sig-alg
  \begin{itemize}
    \item $\varnothing \in (a,b] \Rightarrow \varnothing \in \mathbb{G}_3 $
    \item Suppose $A_j = \cup_{i=1}^{N_j} ((a_i,b_i] \cup (-\infty,b] \cup (a,\infty))$ \\
    So then $\cup_j^\infty A_j = \cup_j^\infty (\cup_{i=1}^{\infty} ((a_i,b_i] \cup (-\infty,b] \cup (a,\infty))) \in \mathbb{G}_3 \qed$
  \end{itemize}
  So, G3 is a sigma algebra
  \end{enumerate}
\end{solution}

\begin{solution} (1.7)

Suppose $\mathbb{A}$ is an algebra. \textbf{WTS:} ${\varnothing,X} \subset \mathcal{A} \subset \mathcal{P}(X)$

\end{solution}

\end{document}
