\documentclass[letterpaper,12pt]{article}
\usepackage{array}
\usepackage{threeparttable}
\usepackage{geometry}
\geometry{letterpaper,tmargin=1in,bmargin=1in,lmargin=1.25in,rmargin=1.25in}
\usepackage{fancyhdr,lastpage}
\pagestyle{fancy}
\lhead{}
\chead{}
\rhead{}
\lfoot{}
\cfoot{}
\rfoot{\footnotesize\textsl{Page \thepage\ of \pageref{LastPage}}}
\renewcommand\headrulewidth{0pt}
\renewcommand\footrulewidth{0pt}
\usepackage[format=hang,font=normalsize,labelfont=bf]{caption}
\usepackage{listings}
\lstset{frame=single,
  showstringspaces=false,
  columns=flexible,
  basicstyle={\small\ttfamily},
  numbers=none,
  breaklines=true,
  breakatwhitespace=true
  tabsize=3
}
\usepackage{amsmath}
\usepackage{amssymb}
\usepackage{amsthm}
\usepackage{harvard}
\usepackage{setspace}
\usepackage{float,color}
\usepackage[pdftex]{graphicx}
\usepackage{hyperref}
\hypersetup{colorlinks,linkcolor=red,urlcolor=blue}
\theoremstyle{definition}
\newtheorem{theorem}{Theorem}
\newtheorem{acknowledgement}[theorem]{Acknowledgement}
\newtheorem{algorithm}[theorem]{Algorithm}
\newtheorem{axiom}[theorem]{Axiom}
\newtheorem{case}[theorem]{Case}
\newtheorem{claim}[theorem]{Claim}
\newtheorem{conclusion}[theorem]{Conclusion}
\newtheorem{condition}[theorem]{Condition}
\newtheorem{conjecture}[theorem]{Conjecture}
\newtheorem{corollary}[theorem]{Corollary}
\newtheorem{criterion}[theorem]{Criterion}
\newtheorem{definition}[theorem]{Definition}
\newtheorem{derivation}{Derivation} % Number derivations on their own
\newtheorem{example}[theorem]{Example}
\newtheorem{exercise}[theorem]{Exercise}
\newtheorem{lemma}[theorem]{Lemma}
\newtheorem{notation}[theorem]{Notation}
\newtheorem{problem}[theorem]{Problem}
\newtheorem{proposition}{Proposition} % Number propositions on their own
\newtheorem{remark}[theorem]{Remark}
\newtheorem{solution}[theorem]{Solution}
\newtheorem{summary}[theorem]{Summary}
%\numberwithin{equation}{section}
\bibliographystyle{aer}
\newcommand\ve{\varepsilon}
\newcommand\boldline{\arrayrulewidth{1pt}\hline}


\begin{document}

\begin{flushleft}
  \textbf{\large{Problem Set \#1}} \\
  Measure Theory, Jan Ertl \\
  Alex Weinberg
\end{flushleft}

\vspace{5mm}

\section{}
\begin{solution}(1.3)
  \begin{enumerate}
  \item
  Suppose $A \in \mathbb{R} $ open. So $A \in \mathbb{G}_1$ by definition. Take $A^c \in \mathbb{R}$ which is closed by properties of open/closed sets. \\
  $\Rightarrow$ $A^c \not \in \mathcal{G}_1$ by definition of $\mathcal{G}_1$ \\
  $\Rightarrow$ G1 not closed under complements. So G1 is not an algebra. $\qed$

  \item \textbf{WTS:} G2 algebra
  \begin{itemize}
  \item $\emptyset \in \mathcal{G}_2 \\
  \Rightarrow \emptyset \in \mathcal{G}_2 $
  \item Suppose $A_j = \bigcup_{i=1}^{N_j} (a_i,b_i] $ \\
  So then $\bigcup_j^M A_j = \bigcup_j^M (\bigcup_{i=1}^{N_j} ((a_i,b_i]) \in \mathcal{G}_2$
  \item Suppose $A_j = \bigcup_{i=1}^{N_j} (a_i,b_i] $ \\
  So then complement $(-\infty, a_1] \bigcup (b_n, \infty) \in \mathcal{G}_2 \qed$
  \end{itemize}
  So G2 is an algebra.

  \item \textbf{WTS:} G3 sig-alg
  \begin{itemize}
    \item $\emptyset \in \mathcal{G}_3$
    \item Suppose $A_j = \cup_{i=1}^{N_j} ((a_i,b_i] \cup (-\infty,b] \cup (a,\infty))$ \\
    So then $\cup_j^\infty A_j = \cup_j^\infty (\cup_{i=1}^{\infty} ((a_i,b_i] \cup (-\infty,b] \cup (a,\infty))) \in \mathbb{G}_3 \qed$
  \end{itemize}
  So, G3 is a sigma algebra
  \end{enumerate}
\end{solution}

\begin{solution} (1.7)
Suppose $\mathcal{A}$ is an sigma algebra. \\
\textbf{WTS:} $\{\emptyset,X\} \subset \mathcal{A} \subset \mathcal{P}(X)$ \\
\textbf{Pf:} $\emptyset \in S$ for every S sigma algebra. Also, S must be closed under complements
therefore $\emptyset^c = X \in S$. So the smallest possible sigma algebra is $\{\emptyset,X\}$. Also, suppose $A \in \mathcal{A}$ so $A \in \mathcal{P}(X)$ because $\mathcal{A} \subset X$ \\$ \therefore \mathcal{A} \subset \mathcal{P}(X)$

\end{solution}

\begin{solution} (1.10) \\
i) $\emptyset \in S_\alpha$  \quad $ \forall \alpha$ by definition of sig-alg. $\Rightarrow \emptyset \in \cap^\alpha S_\alpha$ \\
ii) suppose $A_1,.... \in \cap^\alpha S_\alpha$ this implies $A_1,.... \in S_\alpha \forall \alpha$ \\
So the union of $A_i \in S_\alpha$ for every alpha. \\~\\
So $\cup A_1,... \in \cap^\infty S_\alpha$ \\
Also, suppose $A \in \bigcap S_\alpha$ now we know that $A \in S_\alpha \quad \forall \alpha$ by definition of sigma-algebra. $A^c \in S_\alpha \quad \forall \alpha$ so, $A^c \in \bigcap S_\alpha$
Therefore intersection is a sigma algebra.

\end{solution}

\begin{solution} (1.17) \\
i) We know \[ \mu(A\cup B)) = \mu(A) + \mu(B)\] if \[A\cap B = \emptyset\].
Now suppose $A \subset B$ and
$B = A \cup U$. \\
So, \[\mu(A \cup U) = \mu(A) + \mu(U) \ge \mu(A)\] because measure is valued on positive reals.
\\~\\

ii) We know, \[
\mu (\cup_{i=1}^\infty A_i) = \sum_{i=1}^\infty \mu(A_i) - \mu(\cap_i A_i)
\]
So, \[
\mu(\cup_{i=1}^n A_i) + \underbrace{\mu(\cap_i^\infty A_i)}_{\ge 0} = \sum_{i=1}^\infty \mu(A_i)
\]
\end{solution}

\begin{solution} (1.18) \\
\textbf{WTS:}\[
\lambda(A) = \mu(A\cap B)
\]
\textbf{Pf:}\[ A,B \subset S \Rightarrow (A \cap B) \subset S. \Rightarrow \emptyset \cap B
= \emptyset
\]
So, i) \[ \lambda(\emptyset) = \mu(\emptyset) = 0
\]
And because intersection is in S and , \[\lambda(A) = \mu(A\cap B) \Rightarrow
\lambda(\cup^\infty A_i) = \mu(\cup^\infty (A_i \cap B)) = \sum^\infty \underbrace{\mu(A_i \cap B)}_{\lambda(A_i)}
= \sum^\infty \lambda(A_i)
\]
\end{solution}

\newpage
\begin{solution} (1.20) \\
\begin{gather*}
\mu(A_1) - \mu(\lim_{n \rightarrow \infty} (A_n))\\
= \mu(A_1 \bigcap \lim_{n \rightarrow \infty} A_n) \\
= \mu(\lim_{n \rightarrow \infty} (A_1 \bigcap A_n) ) \\
= \lim_{n \rightarrow \infty} \mu(A_1 \bigcap A_n) \\
= \lim_{n \rightarrow \infty} \mu(A_1) - \mu (A_n)) \\
= \mu(A_1) - \lim_{n \rightarrow \infty} \mu(A_n) \\
\therefore \mu(\lim_{n \rightarrow \infty} (A_n)) = \lim_{n \rightarrow \infty} \mu(A_n)
\end{gather*}
\end{solution}

\section{}
\begin{solution}{(2.10)}
We know from the countable subadditivity of the outer measure that
$$\mu^*(B) \leq \mu^*(B \cap E) + \mu^*(B \cap E^c) $$
So the if $\geq$ from the theorum holds, and $\leq$ holds from definition of outer measure and it is equivalent to replace it with equality.
\end{solution}

\begin{solution}{(2.14)} \textbf{WTS:} $\sigma(\mathcal{O}) \subset \mathcal{M}$
From Caratheodory and construction of lebasque measure as a infinite ocllection of the form $(a,b]$ and $(-\infty, a]$ we know
\begin{align}
\sigma(\mathcal{A})  \subset \mathcal{M} \\
if \quad o \in \sigma(\mathcal{O}) \quad then \quad o \in \sigma(\mathcal{A}) \\
so, \quad o \in \mathcal{M}
\end{align}

\end{solution}

%----------------------------------
\section{}
\begin{solution}{(3.1)}
Suppose $r = \bigcup_{n \in \mathbb{N}} r_i$ \\
So $\mu(\bigcup_{n \in \mathbb{N}}r_i) = \sum_{n \in \mathbb{N}}\mu(r_i)$ \\
but by construction of lebesque measure, $\mu(r_i) = 0, \forall i$\\
So, $\mu(r) = \mu(\bigcup_{n \in \mathbb{N}} r_i) = 0$
\end{solution}

\begin{solution}{(3.4)} \\
Because the set of all measureable sets is a sigma-algebra, it is closed under complements. So those conditions provided each are complements of each other. \\
I'll show that the sets being measurable are all equivalent statements. \\
$f^{-1}((-\infty, a))$ is measurable $\iff f^{-1}([a, \infty))$ is measurable (they are complements). \textbf{WTS:}
$$ f^{-1}((-\infty, a)) \in \mathcal{M} \iff f^{-1}((-\infty, a]) \in \mathcal{M} $$

Left to Right: Suppose sets of the form $f^{-1}((-\infty, a]) \in \mathcal{M}$. \\
Now, we construct a sequence of sets $E_{i,n} = f^{-1}((-\infty, a - \frac{1}{n}]) \in \mathcal{M} $. This countable union $\cup_{n=1}^\infty = f^{-1}((-\infty, a])$ is in $\mathcal{M}$ \\~\\
Right to Left: Suppose sets of the form $f^{-1}((-\infty, a)) \in \mathcal{M}$. Then their complements, sets of the form $f^{-1} ([a, \infty))$ are also $in \mathcal{M}$. We can use a similar argument, employing sets of the form $f^{-1}([ a + \frac{1}{n}, \infty))$ to show that sets of the form $f^{-1}((a, \infty)) \in \mathcal{M}$. This shows that the complements of these sets, $f^{-1}((-\infty, a])$, are also elements of $\mathcal{M}$ \\
\end{solution}


\begin{solution}{(3.7)} \\
\textbf{WTS:}
\begin{enumerate}
	\item $f + g$
	\item $f \cdot g$
	\item $\max(f,g)$
	\item $\min(f,g)$
	\item $|f|$
\end{enumerate}
\textbf{Pf:}
\begin{enumerate}
  \item Take $F(f(x) + g(x)) = f(x) + g(x)$. So $F$ is cont. and (part 4) measurable. So, $f + g$ is measurable.
  \item Take $F(f(x) + g(x)) = f(x)g(x)$. So $F$ is cont. and (part 4) measurable. So, $f \cdot g$ is measurable.
  \item As $f$ and $g$ are measurable functions on $(X,\mathcal{M})$, \\$ \forall a \in \mathbb{R}$, $\{x \in X : f(x) < a \} \in \mathcal{M}$ and $\{x \in X : g(x) < a \} \in \mathcal{M}$. \\
  So,  $$\{x \in X : \max(f(x),g(x)) < a \} = \{x \in X : f(x) < a \} \cap \{x \in X : g(x) < a \}$$. $\mathcal{M}$ is closed under countable intersections, therefore, $\{x \in X : \max(f(x),g(x)) < a \} \in \mathcal{M}$, so that $\max(f(x), g(x))$ is measurable.
  \item $\{x \in X : \min(f(x),g(x)) > a \} = \{x \in X : f(x) > a \} \cap \{x \in X : g(x) > a \}$. $\mathcal{M}$ is closed under countable intersections, \\
  So, $\{x \in X : \min(f(x),g(x)) > a \} \in \mathcal{M}$, so that $\min(f(x), g(x))$ is measurable.Basically the same proof as above.
  \item Because $\{x \in X : |f(x)| > a \} = \{x \in X : f(x) < -a \} \cup \{x \in X : f(x) > a \}$. Both of these sets are in $\mathcal{M}$. $\mathcal{M}$ is closed under countable unions, therefore, $\{x \in X : |f(x)| > a \} \in \mathcal{M}$, so that $|f(x)|$ is measurable.
\end{enumerate}
\end{solution}

\begin{solution}{(3.14)} \\
\textbf{WTS:}
$$ \forall \epsilon > 0, \exists N=N(\epsilon) \text{ such that } n \geq N \implies |f(x)-s_n(x)| < \epsilon, \forall x \in X$$

Let $\epsilon > 0$. Suppose $f(x) < M$. Pick $N > M$ in the naturals. Then $f(x) < N \quad \forall x$  and $x \notin E_\infty^{N}$. We also see that there exists $N_1$ such that
$$
N_1 > N \text{ and } \frac{1}{2^{N_1}} < \epsilon
$$
Now, it follows that for $n>N_1$,
$$
\forall x \in X, x \in E_i^n \text{ for some index } 0 \leq i \leq N_1, i \in \mathbb{N}
$$
Then $f(x) \in [\frac{i-1}{2^n}, \frac{i}{2^n}) $ and our simple function in this interval is $s_n(x) = \frac{i-1}{2^n} $ \\
But because we chose this $N_1$ so that it works for every single $x \in X$, and so $|f(x) - s_n(x)| < \frac{1}{2^n} < \frac{1}{2^{N^2}} < \epsilon $ we have uniform convergence.

\end{solution}

\begin{solution}{(4.13)} \\
\textbf{WTS:} $f \in \mathcal{L}^1(\mu, E)$ \\ Which is to say, \\
\textbf{WTS:} $\int_E f^+ d\mu$ finite \\
\textbf{WTS:} $\int_E f^- d\mu$ finite \\

We are given $||f|| = f^+ + f^-$. \\
$0 \leq f^+$ and $0 \leq f^-$ from the definition \\
Because $||f|| < M$ on $E$, then $0 \leq f^+ < M$ and $0 \leq f^- < M$ on $E$.

Because $\mu(E) < \infty$, we have that,
\begin{align*}
  &\int_E f^+ d\mu < M \mu(E) < \infty \\
  &\int_E f^- d\mu < M \mu(E) < \infty \\
\end{align*}
This is what we needed.\\
$\therefore \quad f \in \mathcal{L}^1(\mu, E)$.
\end{solution}

\begin{solution}{(4.14)} \\
Contrapositive is easier to show. \\

Suppose there exists a measurable set $\hat{E} \subset E$ such that $f$ is infinite on $\hat{E}$. \\WLOG: $f = +\infty$
\begin{equation*}
	\infty = \int_{\hat{E}} f d\mu \leq \int_E f d\mu \leq \int_E ||f|| d\mu
\end{equation*}
But this implies that $f \not\in \mathcal{L}^1(\mu,E)$. Contradiction so our statement is true.
\end{solution}

\begin{solution}{(4.15)}
Suppose $f,g \in \mathcal{L}^1(\mu,E)$. \\
Define the set of simple functions $B(f) = \{ s : 0 \leq s \leq f, s \text{ simple, measurable}  \}$. \\
Suppose WLOG $f \leq g$. If follows that $f^+ \leq g^+$ and $f^- \geq g^-$. \\
So we have that $B(f^+) \subset B(g^+)$ and $B(g^-) \subset B(f^-)$ \\
Which gives us that $ \int_E f^+ d\mu \leq \int_E g^+ d\mu$ and $\int_E f^- d\mu \geq \int_E g^- d\mu$. \\
From definition of the lebesque integral we know,
\begin{equation*}
	\int_E fd\mu = \int_E f^+ d\mu - \int_E f^- d\mu \leq \int_E g^+ d\mu - \int_E g^- d\mu = \int_E g d\mu
\end{equation*}
So,
\begin{equation*}
	\int_E fd\mu \leq \int_E g d\mu
\end{equation*}
\end{solution}

\begin{solution}{(4.16)} \\
Suppose that $f \in \mathcal{L}(E, \mu)$. \\
So, $ \int_E f^+ d\mu < \infty, \int_E f^- d\mu < \infty$. \\
As $A \subset E$, \\
$\int_A f^+ d\mu < \infty$ and $\int_A f^- d\mu < \infty$. \\
So, $f \in \mathcal{L}(A, \mu)$.
\end{solution}

\begin{solution}{(4.21)}
From above, $ \lambda() = \int_{} f d\mu $ is a measure on $\mathcal{M}$. \\
Therefore,
$$ \lambda(A) = \lambda(A \setminus B) + \lambda(A \cap B) = \lambda(A \setminus B) + \lambda(B) $$
So,
$$ \int_A fd\mu = \int_{A \setminus B} fd\mu + \int_B fd\mu = \int_B fd\mu $$

\end{solution}



\end{document}
