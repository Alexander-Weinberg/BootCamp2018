\documentclass[12pt]{amsart}

%Below are some necessary packages for your course.
\usepackage{amsfonts,latexsym,amsthm,amssymb,amsmath,amscd,euscript}
\usepackage{framed}
\usepackage{fullpage}
\usepackage{hyperref}
    \hypersetup{colorlinks=true,citecolor=blue,urlcolor =black,linkbordercolor={1 0 0}}

\newenvironment{problem}[2][Problem]{\begin{trivlist}
\item[\hskip \labelsep {\bfseries #1}\hskip \labelsep {\bfseries #2.}]}{\end{trivlist}}


% You can define new commands to make your life easier.
\newcommand{\R}{\mathbb R}
\newcommand{\C}{\mathbb C}
\newcommand{\F}{\mathbb F}
\newcommand{\Q}{\mathbb Q}
\newcommand{\Z}{\mathbb Z}
\newcommand{\N}{\mathbb N}


%\numberwithin{equation}{section}
\title{Math Pset 3:\\
       Spectral Theory}
\author{Alex Weinberg\\
        OSM Boot Camp 2018}


\begin{document}

\maketitle


\begin{problem}{2} Find the eigenvalues of D[p](x) = p'(x)
\end{problem}
\begin{proof}
    \[
 D =   \begin{bmatrix}
    0 & 1 & 0 \\
    0 & 0 & 2 \\
    0 & 0 & 0 \\
    \end{bmatrix}
    \]
    So, \[ p_A(\lambda) = 
   \begin{bmatrix}
    \lambda & 0 & 0 \\
    0 & \lambda & 0 \\
    0 & 0 & \lambda \\
    \end{bmatrix} -
    \begin{bmatrix}
    0 & 1 & 0 \\
    0 & 0 & 2 \\
    0 & 0 & 0 \\
\end{bmatrix} =
\begin{bmatrix}
    \lambda & -1 & 0 \\
    0 & \lambda & -2 \\
    0 & 0 & \lambda \\
\end{bmatrix}\\
\]
So, \[
det(p_D(\lambda)) = \lambda^3 =0 \iff \lambda=0
\]
\end{proof}

\begin{problem}{4}
\begin{proof}
    Eigenvalues in 2x2 matrix are roots to 
    \[p_A(\lambda) = \lambda^2 + \lambda(a+d) + (ad - bc) = 0
    \]
    Quadratic formula gives us:
\end{proof}
\end{problem}
\end{document}

