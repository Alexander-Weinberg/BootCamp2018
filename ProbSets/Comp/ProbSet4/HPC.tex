\documentclass[12pt]{article}

\usepackage[margin=1in]{geometry}
\usepackage{fancyhdr}
\usepackage{setspace}
\pagestyle{fancy}
\usepackage{amsmath, amsthm, amssymb, amsfonts, mathtools, xfrac,mathrsfs}
\usepackage[utf8]{inputenc}
\usepackage[english]{babel}
\usepackage{graphicx,dsfont}
\usepackage{braket, bm}
\usepackage[dvipsnames]{xcolor}

\everymath{\displaystyle}
\headheight=20pt


\newcommand{\N}{\mathbb{N}}

\usepackage{listings}
\lstset{frame=single,
  language=C++,
  showstringspaces=false,
  columns=flexible,
  basicstyle={\small\ttfamily},
  keywordstyle=\bfseries\color{OliveGreen},
  commentstyle=\itshape\color{purple},
  identifierstyle=\color{blue},
  stringstyle=\color{red},
  numbers=none,
  breaklines=true,
  breakatwhitespace=true
  tabsize=3
}

\newenvironment{problem}[2][Problem]{\begin{trivlist}
\item[\hskip \labelsep {\bfseries #1}\hskip \labelsep {\bfseries #2.}]}{\end{trivlist}}


\title{Homework}
\lhead{High Performance Computing}
\chead{Pset 4}
\rhead{Alex Weinberg}

\begin{document}
%-------------------------------------------------------------------------------
\begin{problem}{1}
I have 100 CPU cores. My code is 0.4 \% serial code. What is maximum speed-up I can obtain?
\end{problem}
\begin{proof}
Speedup from parallel is limited by
$$
S(p,N) = \frac{1}{f + \frac{(1-f)}{p}}
$$
In our question. f = 0.4 and p =100. So,
\begin{align*}
S(100,N) = \frac{1}{0.4 + \frac{(1-0.4)}{100}} \approx 2.46
\end{align*}
\end{proof}
%-------------------------------------------------------------------------------
\begin{problem}{2}
\begin{itemize}
  \item Go to your home directory on MIDWAY (e.g. /home/simonsch)
  \item Create a folder named Exercises day1
  \item in there, create a sub-folder denoted Fortran as well as a sub-folder named CPP
  \item go to your favourite programming language's folder.
  \item Write a little program in that language that reads in your name from the terminal and write “Hello YOURNAME, how are you”.
  \item copy this file to your local laptop/desktop by scp.
  \item compile the code, create an executable that is called “hidiho.exec”.
  \item Hard-code YOURNAME into your *.cpp/*.f90 file.
  \item Adjust a slurm job.sh file such that you can submit the executable in batch mode.
\end{itemize}
\begin{proof}~\\
\begin{itemize}
\item makedir exercises\_day1
\item cd exercises\_day1
\item makedir Fortran
\item makedir CPP
\item scp weinberga@midway1.rcc.uchicago.edu: /home/weinberga/exercises\_day1/CPP/hello\_name.cpp     /Users/alexweinberg/Desktop/BootCamp2018/ProbSets/Comp/ProbSet4
\item g++ hello\_name.cpp -o hidiho.exec
\item \textbf{NEED HELP MAKING SLURM SH FILE}
\end{itemize}
\end{proof}
\end{problem}
\lstinputlisting[caption=Code for Hello-name]{hello_name.cpp}

%-------------------------------------------------------------------------------
\newpage
\begin{problem}{3}
Write a program in Fortran or CPP that reads arbitrary values a, b, c from the terminal (stdin) and prints the solution to the quadratic Equation.\\
$$ax^2 + bx + c=0$$
\end{problem}
\begin{proof}~\\
\lstinputlisting[caption= Quadratic Solver]{quadratic.cpp}
\end{proof}
%-------------------------------------------------------------------------------
%-------------------------------------------------------------------------------
\newpage
\begin{problem}{4}
Compute pi and write a makefile
\end{problem}
\lstinputlisting[caption= Estimate $\pi$]{pi.cpp}
\lstinputlisting[caption= Makefile for pi.cpp]{makefile_pi}
{\huge HOW DO I COMBINE MAKEFILES????}
\lstinputlisting[caption= Submit batch to pi]{submit_pi.sh}

%-------------------------------------------------------------------------------
%-------------------------------------------------------------------------------
\begin{problem}{5}
Compute Pi using Monte Carlo. \\
Experiment with the number of random number you create (N = 100, 1,000, 10,000).\\
Run the code both in interactive as well as in batch mode.
\end{problem}
\lstinputlisting[caption=MonteCarlo estimate pi]{compute_pi.cpp}
{\huge WHY ONLY WORK ON LOCAL, NOT MIDWAY????}

%-------------------------------------------------------------------------------

%-------------------------------------------------------------------------------
\begin{problem}{6}
Run the project in midway
\end{problem}
\begin{proof}
Hello
\end{proof}
%-------------------------------------------------------------------------------
%-------------------------------------------------------------------------------
\end{document}
